\documentclass[12pt]{article}
\usepackage[utf8]{inputenc}
\usepackage{geometry}
\geometry{
    a4paper,
    left = 3cm,
    right = 2.5cm
}
\usepackage{abstract}
\renewcommand{\abstractnamefont}{\normalfont\large\bfseries}
\renewcommand{\abstracttextfont}{\normalfont\normalsize}

\title{\textbf{\Large PolSAR image analysis based on Geodesic Distance}}
\author{\large
Danilo Fernandes \\ \textit{\large Universidade Federal de Alagoas}
}
\date{}

\begin{document}

\maketitle

\begin{abstract}

Polarimetric Syntetic Aperture Radar -- PolSAR -- is a remote sensing device that obtains earthly surface images which describe the dependence of scattering properties of the surface on the polarization of the used electromagnetic waves. 
The data that compose such images are commonly represented by a scattering or covariance matrix, where both are complex. 
An interesting approach to analyze these data is transforming them into a Kennaugh matrix, which is real and preserves the backscatter information.
Once in this representation, it is possible to measure the Geodesic Distance to elementary backscatterers, i.e., between data and prototypes. 
In our research, we discovered that distances to elementary backscatterers on specific regions obey statistical models. 
In this way, it is possible classify regions, like forests, in huge PolSAR images based on statistical behavior of the distances.
\end{abstract}

\end{document}

