\documentclass[12pt]{article}
\usepackage[utf8]{inputenc}
\usepackage{geometry}
\geometry{
    a4paper,
    left = 3cm,
    right = 2.5cm
}
\usepackage{abstract}
\renewcommand{\abstractnamefont}{\normalfont\large\bfseries}
\renewcommand{\abstracttextfont}{\normalfont\normalsize}

\title{\textbf{\Large PolSAR image analysis based on Geodesic Distance}}
\author{\large
Danilo Fernandes \\ \textit{\large Universidade Federal de Alagoas}
}
\date{}

\begin{document}

\maketitle

\begin{abstract}

Polarimetric Syntetic Aperture Radar -- PolSAR -- is a device used on remote sensing that obtains earthly suface images which describe the dependence of scattering properties of the surface on the polarization of the used eletromagnetic waves. The data that compose this images is commomly represented by a scattering or covariance matrix, where both are complex. A interesting approach to analyse this data is transform them in a Kennaugh matrix, which is real and preserves the backscatter information, and measure the Geodesic Distance to elementary backscatterers. In our research, we discover that distances to elementary backscatterers on specific regions obey statistical models. In this way, it's possible classify regions, like forests, in huge PolSAR images based on statistical behavior of the distances of its data to known backscatterers.
\end{abstract}

\end{document}

