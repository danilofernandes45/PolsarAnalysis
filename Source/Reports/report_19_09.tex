\documentclass[12pt]{article}

\usepackage[portuguese]{babel}
\usepackage[utf8]{inputenc}
\usepackage{amsmath}
\usepackage{commath}
\usepackage[alf]{abntex2cite}
\usepackage{indentfirst}
\usepackage{graphicx}
\usepackage{multicol,lipsum}
\usepackage{subfig}
\usepackage{geometry}
\usepackage[alf]{abntex2cite}
\usepackage{subfig}

\geometry{
  paper = a4paper,
  inner = 3cm,
  outer = 3cm,
  top = 2cm,
  bottom = 2cm
}

\begin{document}
%\maketitle

\onehalfspacing

\begin{titlepage}
\begin{center}

\Huge{Universidade Federal de Alagoas}\\
\large{Instituto de Computação}\\ 
\large{Laboratório de Computação Científica e Análise Numérica}\\ 
\vspace{220pt}
\textbf{\LARGE{Research report}}\\
%\title{{\large{Título}}}
\vspace{3,5cm}
\end{center}

\begin{flushleft}
\begin{tabbing}
Student: Danilo Fernandes Costa\\
Professor: Alejandro Frery\\
\end{tabbing}
\end{flushleft}
\vspace{1cm}

\begin{center}
\vspace{\fill}
September\\
2019
\end{center}
\end{titlepage}

\section{Introduction}

In this report, some results of the analysis of samples referring to plantation regions observed over time are shown. In total, five observations were made with a time interval of 25 days between them.
These regions consist of three soybeans crops, three wheat, two oats and two canola.


\end{document}