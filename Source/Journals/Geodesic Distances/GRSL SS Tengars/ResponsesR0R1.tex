\RequirePackage{xr}

\documentclass[journal,onecolumn,draftcls,11pt]{IEEEtran}

\usepackage{graphicx}
\graphicspath{{../../../../Figures/GRSL_2020/}{../../../../Figures/GRSL_2020/FactorPlots/}{../../../../Images/GRSL_2020/}{../../../../Figures/Soybeans_231/}}

\usepackage{subcaption}
\captionsetup[table]{font=small,size=smaller,textfont=sc}
\captionsetup[figure]{font=small,size=smaller}

\usepackage{booktabs}
\usepackage[T1]{fontenc}
\usepackage{cite}
\usepackage[cmex10]{amsmath}
\usepackage{color}
\usepackage{bm,bbm}
\usepackage{wasysym}
\usepackage{texnames}

\externaldocument{StatisticalGeodesicFeaturesR1}
\usepackage[listings]{tcolorbox}

\usepackage{siunitx}
\DeclareSIUnit\pertenmille{\text{\textpertenthousand}}
\usepackage{multirow,bigstrut}

\DeclareMathOperator{\Tr}{Tr}

%DIF PREAMBLE EXTENSION ADDED BY LATEXDIFF
%DIF UNDERLINE PREAMBLE %DIF PREAMBLE
\RequirePackage[normalem]{ulem} %DIF PREAMBLE
\RequirePackage{color}\definecolor{RED}{rgb}{1,0,0}\definecolor{BLUE}{rgb}{0,0,1} %DIF PREAMBLE
\providecommand{\DIFadd}[1]{{\protect\color{blue}\uwave{#1}}} %DIF PREAMBLE
\providecommand{\DIFdel}[1]{{\protect\color{red}\sout{#1}}}                      %DIF PREAMBLE
%DIF SAFE PREAMBLE %DIF PREAMBLE
\providecommand{\DIFaddbegin}{} %DIF PREAMBLE
\providecommand{\DIFaddend}{} %DIF PREAMBLE
\providecommand{\DIFdelbegin}{} %DIF PREAMBLE
\providecommand{\DIFdelend}{} %DIF PREAMBLE
\providecommand{\DIFmodbegin}{} %DIF PREAMBLE
\providecommand{\DIFmodend}{} %DIF PREAMBLE
%DIF FLOATSAFE PREAMBLE %DIF PREAMBLE
\providecommand{\DIFaddFL}[1]{\DIFadd{#1}} %DIF PREAMBLE
\providecommand{\DIFdelFL}[1]{\DIFdel{#1}} %DIF PREAMBLE
\providecommand{\DIFaddbeginFL}{} %DIF PREAMBLE
\providecommand{\DIFaddendFL}{} %DIF PREAMBLE
\providecommand{\DIFdelbeginFL}{} %DIF PREAMBLE
\providecommand{\DIFdelendFL}{} %DIF PREAMBLE
\newcommand{\DIFscaledelfig}{0.5}
%DIF HIGHLIGHTGRAPHICS PREAMBLE %DIF PREAMBLE
\RequirePackage{settobox} %DIF PREAMBLE
\RequirePackage{letltxmacro} %DIF PREAMBLE
\newsavebox{\DIFdelgraphicsbox} %DIF PREAMBLE
\newlength{\DIFdelgraphicswidth} %DIF PREAMBLE
\newlength{\DIFdelgraphicsheight} %DIF PREAMBLE
% store original definition of \includegraphics %DIF PREAMBLE
\LetLtxMacro{\DIFOincludegraphics}{\includegraphics} %DIF PREAMBLE
\newcommand{\DIFaddincludegraphics}[2][]{{\color{blue}\fbox{\DIFOincludegraphics[#1]{#2}}}} %DIF PREAMBLE
\newcommand{\DIFdelincludegraphics}[2][]{% %DIF PREAMBLE
	\sbox{\DIFdelgraphicsbox}{\DIFOincludegraphics[#1]{#2}}% %DIF PREAMBLE
	\settoboxwidth{\DIFdelgraphicswidth}{\DIFdelgraphicsbox} %DIF PREAMBLE
	\settoboxtotalheight{\DIFdelgraphicsheight}{\DIFdelgraphicsbox} %DIF PREAMBLE
	\scalebox{\DIFscaledelfig}{% %DIF PREAMBLE
		\parbox[b]{\DIFdelgraphicswidth}{\usebox{\DIFdelgraphicsbox}\\[-\baselineskip] \rule{\DIFdelgraphicswidth}{0em}}\llap{\resizebox{\DIFdelgraphicswidth}{\DIFdelgraphicsheight}{% %DIF PREAMBLE
				\setlength{\unitlength}{\DIFdelgraphicswidth}% %DIF PREAMBLE
				\begin{picture}(1,1)% %DIF PREAMBLE
				\thicklines\linethickness{2pt} %DIF PREAMBLE
				{\color[rgb]{1,0,0}\put(0,0){\framebox(1,1){}}}% %DIF PREAMBLE
				{\color[rgb]{1,0,0}\put(0,0){\line( 1,1){1}}}% %DIF PREAMBLE
				{\color[rgb]{1,0,0}\put(0,1){\line(1,-1){1}}}% %DIF PREAMBLE
				\end{picture}% %DIF PREAMBLE
			}\hspace*{3pt}}} %DIF PREAMBLE
} %DIF PREAMBLE
\LetLtxMacro{\DIFOaddbegin}{\DIFaddbegin} %DIF PREAMBLE
\LetLtxMacro{\DIFOaddend}{\DIFaddend} %DIF PREAMBLE
\LetLtxMacro{\DIFOdelbegin}{\DIFdelbegin} %DIF PREAMBLE
\LetLtxMacro{\DIFOdelend}{\DIFdelend} %DIF PREAMBLE
\DeclareRobustCommand{\DIFaddbegin}{\DIFOaddbegin \let\includegraphics\DIFaddincludegraphics} %DIF PREAMBLE
\DeclareRobustCommand{\DIFaddend}{\DIFOaddend \let\includegraphics\DIFOincludegraphics} %DIF PREAMBLE
\DeclareRobustCommand{\DIFdelbegin}{\DIFOdelbegin \let\includegraphics\DIFdelincludegraphics} %DIF PREAMBLE
\DeclareRobustCommand{\DIFdelend}{\DIFOaddend \let\includegraphics\DIFOincludegraphics} %DIF PREAMBLE
\LetLtxMacro{\DIFOaddbeginFL}{\DIFaddbeginFL} %DIF PREAMBLE
\LetLtxMacro{\DIFOaddendFL}{\DIFaddendFL} %DIF PREAMBLE
\LetLtxMacro{\DIFOdelbeginFL}{\DIFdelbeginFL} %DIF PREAMBLE
\LetLtxMacro{\DIFOdelendFL}{\DIFdelendFL} %DIF PREAMBLE
\DeclareRobustCommand{\DIFaddbeginFL}{\DIFOaddbeginFL \let\includegraphics\DIFaddincludegraphics} %DIF PREAMBLE
\DeclareRobustCommand{\DIFaddendFL}{\DIFOaddendFL \let\includegraphics\DIFOincludegraphics} %DIF PREAMBLE
\DeclareRobustCommand{\DIFdelbeginFL}{\DIFOdelbeginFL \let\includegraphics\DIFdelincludegraphics} %DIF PREAMBLE
\DeclareRobustCommand{\DIFdelendFL}{\DIFOaddendFL \let\includegraphics\DIFOincludegraphics} %DIF PREAMBLE
%DIF LISTINGS PREAMBLE %DIF PREAMBLE
\RequirePackage{listings} %DIF PREAMBLE
\RequirePackage{color} %DIF PREAMBLE
\lstdefinelanguage{DIFcode}{ %DIF PREAMBLE
	%DIF DIFCODE_UNDERLINE %DIF PREAMBLE
	moredelim=[il][\color{red}\sout]{\%DIF\ <\ }, %DIF PREAMBLE
	moredelim=[il][\color{blue}\uwave]{\%DIF\ >\ } %DIF PREAMBLE
} %DIF PREAMBLE
\lstdefinestyle{DIFverbatimstyle}{ %DIF PREAMBLE
	language=DIFcode, %DIF PREAMBLE
	basicstyle=\ttfamily, %DIF PREAMBLE
	columns=fullflexible, %DIF PREAMBLE
	keepspaces=true %DIF PREAMBLE
} %DIF PREAMBLE
\lstnewenvironment{DIFverbatim}{\lstset{style=DIFverbatimstyle}}{} %DIF PREAMBLE
\lstnewenvironment{DIFverbatim*}{\lstset{style=DIFverbatimstyle,showspaces=true}}{} %DIF PREAMBLE
%DIF END PREAMBLE EXTENSION ADDED BY LATEXDIFF

\begin{document}

\title{Statistical Properties of Geodesic Roll-Invariant Indexes in PolSAR Data over Crops\\
Revision R1}

\author{Danilo~Fernandes,
	Debanshu~Ratha,
	Avik~Bhattacharya,~\IEEEmembership{Senior~Member,~IEEE},
	and~Alejandro~C.~Frery,~\IEEEmembership{Senior~Member,~IEEE}}

\markboth{IEEE Geoscience and Remote Sensing Letters}%
{D.\ Fernandes et al.\MakeLowercase{\textit{et al.}}: Statistics Geodesic Distances}

\maketitle

\IEEEpeerreviewmaketitle

\section{Editor-in-Chief}

Your paper GRSL-00474-2020 Statistical Properties of Geodesic Roll-Invariant Indexes in PolSAR Data over Crops has been carefully reviewed by the GRSL review panel and found to be unacceptable in its present form. The reviewers did suggest, however, that if completely revised the paper might be found acceptable. We encourage you to revise and resubmit this manuscript as a new paper to GRSL.


\section{Associate Editor}

Both reviewers suggest many weaknesses and inconsistencies that need to be corrected.
Importantly, the overall motivation needs to be clearer and targeted towards the geosciences and remote sensing audience. For example, why is roll-invariance useful, and why is it important to know their statistical distributions.
The overall quality needs to be significantly lifted, with better referencing of key terms and more careful checking of important terms and symbols, among other noted comments.
I recommend to reject with an invitation to resubmit a much improved paper. Please consider whether a 5 page letter may be too restrictive for the revised work and whether TGRS or JSTARS may be an appropriate avenue for resubmission.


\section{Reviewer \#1}

Overall, the manuscript provides a good description of empirical statistical properties of geodesic roll-invariant, polarimetric parameters. There are however a number of apparent inconsistencies to correct and clarifications that should be made prior to publication. One concern I have is whether the required revisions can be accomplished in a Letter format. That concern I leave to the authors and Editor.  
My comments that need to be addressed prior to publication generally follow their appearance in the Letter:

\vskip3em\begin{tcolorbox}[colback=red!5!white,colframe=red!75!black,title=Comment \#1]
I found no reference to [9] in the Letter.
\end{tcolorbox}

We are sorry for this.
We will certify to run \BibTeX\ before submitting the revised version and, with this, no extra references will be added.

\vskip3em\begin{tcolorbox}[colback=red!5!white,colframe=red!75!black,title=Comment \#2]
Named techniques, e.g. Hellinger, Freedman-Diaconis, etc., are not referenced; they should be.
\end{tcolorbox}

The manuscript has been completely reformulated in terms of methodology and results.


\vskip3em\begin{tcolorbox}[colback=red!5!white,colframe=red!75!black,title=Comment \#3]
The last sentence in the paragraph containing eq. (3) is unnecessary (and wrong). GD(K,L) is normalized [0,1]; it is not an "angle".
\end{tcolorbox}

Agreed. We deleted that sentence.


\vskip3em\begin{tcolorbox}[colback=red!5!white,colframe=red!75!black,title=Comment \#4]
Table I is difficult to read. Tabulations of matrices are more accessible in matrix form.	
\end{tcolorbox}

Agreed.
We changed the way of presenting the Kennaugh matrices, 
and limited them to those used in the work.


\vskip3em\begin{tcolorbox}[colback=red!5!white,colframe=red!75!black,title=Comment \#5]
Eq. (4), the purity index is wrongly normalized. An aside: Why is the purity index the square of GD? Wouldn't a linear relationship work just as well?	
\end{tcolorbox}

We use the Geodesic Purity index as defined in Ref.~\cite[Eq.~(33)]{APolSARScatteringPowerFactorizationFrameworkandNovelRollInvariantParametersBasedUnsupervisedClassificationSchemeUsingaGeodesicDistanceinpress}.
Its scale and quadratic form are set to be comparable with other forms of polarimetric purity.

\vskip3em\begin{tcolorbox}[colback=red!5!white,colframe=red!75!black,title=Comment \#6]
The last paragraph in Section II refers to "scattering type angle" and "purity index" but the equations are for "scattering type angle" and "helicity parameter".	
\end{tcolorbox}

Agreed.
We rephrased the paragraph.
Please notice that in this new version we analyze the roll-invariant parameters in their natural domain, without introducing any scaling or transformation.

\vskip3em\begin{tcolorbox}[colback=red!5!white,colframe=red!75!black,title=Comment \#7]
Bottom of the page, the sentence beginning with "Beta random variables..." needs grammatical editing.	
\end{tcolorbox}

That paragraph has changed as a result of the new methodology.


\vskip3em\begin{tcolorbox}[colback=red!5!white,colframe=red!75!black,title=Comment \#8]
Fig. 1 did not convey much information to me. An Overview image of the area with fields highlighted would be useful, then the temporal development of the fields has context. 	
\end{tcolorbox}

We had to stick to these small samples because of the five-page constraint.

\vskip3em\begin{tcolorbox}[colback=red!5!white,colframe=red!75!black,title=Comment \#9]
Fig. 2 \& others: Captions, labels general text must be legible. The fonts used are just too small. 	
\end{tcolorbox}

The new plots have larger captions and labels.


\vskip3em\begin{tcolorbox}[colback=red!5!white,colframe=red!75!black,title=Comment \#10]
The first full paragraph below eq. (8) is not consistent with the definition of scattering type angle given in Section II.B. The Scattering Type is defined as a distance from the ideal trihedral - not some down-selection of pixels based on relative closeness to a set of canonical scattering types.
\end{tcolorbox}

We agree, and we now present the analysis of data fully compliant with the definitions.

\vskip3em\begin{tcolorbox}[colback=red!5!white,colframe=red!75!black,title=Comment \#11]
The discussion of Table IV states that "five out of twenty samples .. below 1", but the Table shows 7 such samples.	
\end{tcolorbox}

We removed the $p$-values, since we now present a different kind of evidence towards the models: the location of the square of the skewness and the kurtosis in the Pearson System plane.

\vskip3em\begin{tcolorbox}[colback=red!5!white,colframe=red!75!black,title=Comment \#12]
Section IV finds that the statistical description of Soy Bean purity index can not distinguish between the July 27 \& the August 20 data. However, looking at the tabulated values in Table II, shows that July 3 \& August 20 fits have parameters that are within ~0.001, which naively are nearly identical. They are much closer to each other than the July 27 \& August 20 parameters are. Please explain.	
\end{tcolorbox}

The new methodology employs the Hyperbolic distribution.
As this law has four parameters, we plot the two principal components, and show that, in particular, the second is able to provide separation evidence among samples which appear close in the $(\widehat p,\widehat q)$ space of scattering type angle and scattering helicity.


\vskip3em\begin{tcolorbox}[colback=red!5!white,colframe=red!75!black,title=Comment \#13]
Fig. 4 discussion, it appears that both wheat and oats equally span the parameter range, not just wheat as mentioned in the paragraph.	
\end{tcolorbox}

As per request of the other reviewer, we are now using new (uncensored) samples.
This has changed the temporal analysis.

\vskip3em\begin{tcolorbox}[colback=red!5!white,colframe=red!75!black,title=Comment \#14]
The Plant Area Index (PAI) is mentioned in the discussion of biomass and temporal development of the crops. It is never defined. I realize that the PAI is the subject of a publication in press, but if you rely on PAI for your analysis it must be defined here. General terms like the relative "biomass" are appropriately inferred from the ground truth / crop reports.	
\end{tcolorbox}

We now provide the definition and a reference, c.f.\ Ref~\cite{TheArchitectureofaDeciduousForestCanopyinEasternTennessee}.

\vskip3em\begin{tcolorbox}[colback=red!5!white,colframe=red!75!black,title=Comment \#15]
The coloured dots in Fig. 4 are hard to follow. A larger font would definitely help. Another approach would colour-code the dots by date, or possibly connect the dots by lines (though that may be distracting). 	
\end{tcolorbox}

We hope the new plots are easier to follow.

\section{Reviewer \#2}
\vskip3em\begin{tcolorbox}[colback=red!5!white,colframe=red!75!black,title=Comment \#1]
\includegraphics[width=\linewidth]{./ReviewR0/R2Q1}
\end{tcolorbox}

Thank you very much for this insightful comment.
We are, indeed, working on this line of research.
So far, our attempts have not been successful, as the literature is scarce on distributions on high-dimensional manifolds other than very simple ones.
Nevertheless, we have reduced the empirical component of our choices.
We now start by the first result on this topic (Ref.~\cite{StatisticalPropertiesofGeodesicDistancesBetweenSamplesandElementaryScatterersinPolSARImagery2019}), who verified (empirically!) that the Beta distribution is a good model for distances between samples and prototypical backscatterers.
This result led us to assume that the Beta law is an acceptable starting point for the scattering type angle.
We then used the Pearson System of distributions, which includes the Beta law and has a nice graphical way of checking the model adequacy, and verified that (i)~both $\alpha_{\text{GD}}$ and~$\tau_{\text{GD}}$ (without scaling or transforming the features) can be explained by the Beta distribution, and (ii)~that $P_{\text{GD}}$ varies across four distributions of this family.
By noticing that there is evidence towards the need of a four-parameter distribution for $P_{\text{GD}}$, we used the Hyperbolic distribution with very good results.

\vskip3em\begin{tcolorbox}[colback=red!5!white,colframe=red!75!black,title=Comment \#2]
\includegraphics[width=\linewidth]{./ReviewR0/R2Q2}
\end{tcolorbox}

With the new methodology we did not need to alter the geodesic roll-invariant features.
They are now analyzed in their natural spaces, and in full compliance of their definitions.

\vskip3em\begin{tcolorbox}[colback=red!5!white,colframe=red!75!black,title=Comment \#3]
\includegraphics[width=\linewidth]{./ReviewR0/R2Q3}
\end{tcolorbox}

We no longer use the LogNormal distribution, but the Hyperbolic law.
Although this distribution has real support, we verified that there is no need to scale the data or to use truncated models.
We did not report these results in order to keep the flow of the paper.

\vskip3em\begin{tcolorbox}[colback=red!5!white,colframe=red!75!black,title=Comment \#4]
\includegraphics[width=\linewidth]{./ReviewR0/R2Q4}
\end{tcolorbox}

We no longer report $p$-values of goodness-of-fit tests.
We rely on the graphical evidence provided by the Pearson System plane, and on the quality of the fit of the Hyperbolic distribution to the most challenging data set.

\vskip3em\begin{tcolorbox}[colback=red!5!white,colframe=red!75!black,title=Comment \#5]
\includegraphics[width=\linewidth]{./ReviewR0/R2Q5}
\end{tcolorbox}

The new experimental part uses the (new) estimates of the parameters of the Beta distribution, and the two first principal components of the parameters of the Hyperbolic distribution.
We obtain an acceptable characterization by considering all the evidence.

\vskip3em\begin{tcolorbox}[colback=red!5!white,colframe=red!75!black,title=Comment \#5]
	\includegraphics[width=\linewidth]{./ReviewR0/R2Q6}
\end{tcolorbox}

The new version is changed to its core.
We no longer comment about the computational platform.

\vskip3em\begin{tcolorbox}[colback=red!5!white,colframe=red!75!black,title=Final Comment]
	\includegraphics[width=\linewidth]{./ReviewR0/R2Q7}
\end{tcolorbox}

We hope the new version is up to the reviewer's standard for GRSL.
We take the opportunity to highlight that the Pearson System of distributions and the Hyperbolic distribution seldom appear on the Remote Sensing literature.
We expect that this paper brings to them the attention they deserve by our community.

\bibliographystyle{IEEEtran}
\bibliography{references}

\end{document}