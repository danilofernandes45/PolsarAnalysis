\documentclass[journal]{IEEEtran}

\usepackage{graphicx}
\graphicspath{{../../../../Figures/GRSL_2020/}{../../../../Images/GRSL_2020/}}

\usepackage[font=footnotesize]{subcaption}
\usepackage{booktabs}

\usepackage{cite}
\usepackage[cmex10]{amsmath}
\usepackage{dblfloatfix}
\usepackage{url}
\usepackage{color}

\begin{document}

\title{Statistical Properties of Geodesic Features in PolSAR Data}

\author{Danilo~Fernandes,
        Debanshu~Ratha,
        Avik~Bhattacharya,~\IEEEmembership{Senior~Member,~IEEE},
        and~Alejandro~C.~Frery,~\IEEEmembership{Senior~Member,~IEEE}% <-this % stops a space
\thanks{D.\ Fernandes is with the}% <-this % stops a space
\thanks{D.\ Ratha and A. Bhattacharya are with the.}% <-this % stops a space
\thanks{A.\ C.\ Frery is with the \textit{Laborat\'orio de Computa\c c\~ao Cient\'ifica e An\'alise Num\'erica} -- LaCCAN, 
	Universidade Federal de Alagoas -- Ufal, 
	57072-900 Macei\'o, AL -- Brazil, and the Key Lab of Intelligent Perception and Image Understanding of the Ministry of Education, Xidian University, Xi'an, China. Email: acfrery@laccan.ufal.br}
\thanks{Manuscript received XX; revised YY.}}

\markboth{IEEE Geoscience and Remote Sensing Letters}%
{D.\ Fernandes et al.\MakeLowercase{\textit{et al.}}: Statistics Geodesic Distances}

\maketitle

\begin{abstract}
The abstract goes here.
\end{abstract}

\begin{IEEEkeywords}
Geodesic distance, PolSAR, Kennaugh representation
\end{IEEEkeywords}

\IEEEpeerreviewmaketitle



\section{Introduction}

\IEEEPARstart{G}{eodesic distances} between points in the Kennaugh representation
No statistical assessment of the properties of such distances under the hypothesis of good samples



In face of this new decomposition theorem, this paper makes qualitative and quantitative analyses of geodesic features (distances, $\alpha$, $\tau$, and purity).
We compute those features between observations and a set of arbitrary prototypes (elementary backscatterers and other representative points).

\section{Metodology}

\subsection{The Kennaugh Representation}

including prototypes
\textcolor{red}{define prototypes}

\subsection{Samples}

We analyse five PolSAR images of the same soilbeans crop region obtained over time, which were made available by Prof. Avik Bhattacharya and his research group. The first image was taken 16 May 2016, followed by four others at time intervals of $24$ days. The images are $30 \times 65$ pixels.
Figures~\ref{fig:day_0} to~\ref{fig:day_96} show the data in Pauli Decomposition. 

\begin{figure}[hbt]
  \centering
  \subcaptionbox{16 May\label{fig:day_0}}{\includegraphics[width = .19\linewidth]{sb231_day_0}}
  \subcaptionbox{9 June\label{fig:day_24}}{\includegraphics[width = .19\linewidth]{sb231_day_24}}
  \subcaptionbox{3 July\label{fig:day_48}}{\includegraphics[width = .19\linewidth]{sb231_day_48}}
  \subcaptionbox{27 July\label{fig:day_72}}{\includegraphics[width = .19\linewidth]{sb231_day_72}}
  \subcaptionbox{20 Aug.\label{fig:day_96}}{\includegraphics[width = .19\linewidth]{sb231_day_96}}
  \caption{Crop samples taken in 2016 over time, in Pauli decomposition}
  \label{fig:sample_images}
\end{figure}

The roadmap for the data analysis of these images is:
\begin{enumerate}
  \item Obtaining the geodesic purity index and scattering type angle of the data;
  \item Making a descriptive analysis of the data with histograms and boxplots;
  \item Fitting the data with models;
  \item Making separability tests.
\end{enumerate}

\subsection{Geodesic Features}

\subsubsection{Geodesic Purity Index}

% A first inspection of the purity indexes suggested that Beta distributions may be a good explanatory model.
% The data look crammed in their original space, though.
% 
% After this initial qualitative analysis, we decided to transform the data applying the logarithmic function.
% Fig.~\ref{fig:histograms_purity_sb231} shows the histograms and boxplots. 
% The closeness to Gaussian distributions is remarkable.
% The QQ-Plots shown in Fig.~\ref{fig:qqplots} provide more evidence of such good adherence.
% Table~\ref{tab:pvalues_purities_sb231} provides the $p$-values of the Shapiro-Wilk test of goodness-of-fit to the Gaussian distribution.
% There is no evidence, thus, to reject the hypothesis that the log-transformed purity data follows Gaussian distributions.

% \begin{figure}[hbt]
%     \centering
%     \subcaptionbox{16 May 2016}{\includegraphics[width = \linewidth]{Figures/Soybeans_231/log_purity_sb231_1}}
%     \subcaptionbox{09 June 2016}{\includegraphics[width = \linewidth]{Figures/Soybeans_231/log_purity_sb231_2}}
%     \subcaptionbox{03 July 2016}{\includegraphics[width = \linewidth]{Figures/Soybeans_231/log_purity_sb231_3}}
%     \subcaptionbox{27 July 2016}{\includegraphics[width = \linewidth]{Figures/Soybeans_231/log_purity_sb231_4}}
%     \subcaptionbox{20 August 2016}{\includegraphics[width = \linewidth]{Figures/Soybeans_231/log_purity_sb231_5}}
%     \caption{Histograms of the $log10$ of geodesic purity index from Soybeans 231}
%     \label{fig:histograms_purity_sb231}
% \end{figure}

% \begin{figure}[hbt]
%     \centering
%     \subcaptionbox{16 May 2016}{\includegraphics[width = \linewidth]{Figures/Canola_43/log_purity_cn43_1}}
%     \subcaptionbox{09 June 2016}{\includegraphics[width = \linewidth]{Figures/Canola_43/log_purity_cn43_2}}
%     \subcaptionbox{03 July 2016}{\includegraphics[width = \linewidth]{Figures/Canola_43/log_purity_cn43_3}}
%     \subcaptionbox{27 July 2016}{\includegraphics[width = \linewidth]{Figures/Canola_43/log_purity_cn43_4}}
%     \subcaptionbox{20 August 2016}{\includegraphics[width = \linewidth]{Figures/Canola_43/log_purity_cn43_5}}
%     \caption{Histograms of the $log10$ of geodesic purity index from Canola 43}
%     \label{fig:histograms_purity_cn43}
% \end{figure}

% \begin{figure}[hbt]
%     \centering
%     \subcaptionbox{16 May 2016}{\includegraphics[width = \linewidth]{Figures/Wheat_104/log_purity_wt104_1}}
%     \subcaptionbox{09 June 2016}{\includegraphics[width = \linewidth]{Figures/Wheat_104/log_purity_wt104_2}}
%     \subcaptionbox{03 July 2016}{\includegraphics[width = \linewidth]{Figures/Wheat_104/log_purity_wt104_3}}
%     \subcaptionbox{27 July 2016}{\includegraphics[width = \linewidth]{Figures/Wheat_104/log_purity_wt104_4}}
%     \subcaptionbox{20 August 2016}{\includegraphics[width = \linewidth]{Figures/Wheat_104/log_purity_wt104_5}}
%     \caption{Histograms of the $log10$ of geodesic purity index from Wheat 104}
%     \label{fig:histograms_purity_wt104}
% \end{figure}

% \begin{figure}[hbt]
%     \centering
%     \subcaptionbox{16 May 2016}{\includegraphics[width = \linewidth]{Figures/Oats_102/log_purity_ot102_1}}
%     \subcaptionbox{09 June 2016}{\includegraphics[width = \linewidth]{Figures/Oats_102/log_purity_ot102_2}}
%     \subcaptionbox{03 July 2016}{\includegraphics[width = \linewidth]{Figures/Oats_102/log_purity_ot102_3}}
%     \subcaptionbox{27 July 2016}{\includegraphics[width = \linewidth]{Figures/Oats_102/log_purity_ot102_4}}
%     \subcaptionbox{20 August 2016}{\includegraphics[width = \linewidth]{Figures/Oats_102/log_purity_ot102_5}}
%     \caption{Histograms of the $log10$ of geodesic purity index from Oats 102}
%     \label{fig:histograms_purity_ot102}
% \end{figure}

%\begin{figure}[hbt]
%  \centering
%  \subcaptionbox{Histograms\label{fig:histograms_purity}}{\includegraphics[width = .95\linewidth]{histograms}}
%  \subcaptionbox{QQPlots\label{fig:qqplots}}{\includegraphics[width = .95\linewidth]{qqplots}}
%  \caption{Descriptive analysis of the logarithm purity values for each image}
%  \label{fig:desc_analysis}
%\end{figure}

% \begin{table}[hbt]
%   \centering
%   \caption{$p$-values from Shapiro-Wilk Test for Soybeans 231}
%   \label{tab:pvalues_purities_sb231}
%   \begin{tabular}{lrrrrr}
%     \toprule
%     \textbf{Day} & \textbf{16 May} & \textbf{09 June} & \textbf{03 July} & \textbf{27 July} & \textbf{20 Aug.}\\
%                  & \textbf{2016} & \textbf{2016} & \textbf{2016} & \textbf{2016} & \textbf{2016}\\\midrule

%     \textbf{$p$-value} & 0.4963 & 0.0650 & 0.3494 & 0.0585 & 0.3919\\
%     \bottomrule
%   \end{tabular}
% \end{table}

% \begin{table}[hbt]
%   \centering
%   \caption{$p$-values from Shapiro-Wilk Test for Canola 43}
%   \label{tab:pvalues_purities_sb231}
%   \begin{tabular}{lrrrrr}
%     \toprule
%     \textbf{Day} & \textbf{16 May} & \textbf{09 June} & \textbf{03 July} & \textbf{27 July} & \textbf{20 Aug.}\\
%                  & \textbf{2016} & \textbf{2016} & \textbf{2016} & \textbf{2016} & \textbf{2016}\\\midrule

%     \textbf{$p$-value} & 0.1143 & 0.7359 & 0.5855 & $2.6\times 10^{-16}$ & $2.6\times 10^{-16}$\\
%     \bottomrule
%   \end{tabular}
% \end{table}

% \begin{table}[hbt]
%   \centering
%   \caption{$p$-values from Shapiro-Wilk Test for Wheat 104}
%   \label{tab:pvalues_purities_sb231}
%   \begin{tabular}{lrrrrr}
%     \toprule
%     \textbf{Day} & \textbf{16 May} & \textbf{09 June} & \textbf{03 July} & \textbf{27 July} & \textbf{20 Aug.}\\
%                  & \textbf{2016} & \textbf{2016} & \textbf{2016} & \textbf{2016} & \textbf{2016}\\\midrule

%     \textbf{$p$-value} & 0.8189 & 0.9042 & 0.0025 & 0.3929 & 0.0544\\
%     \bottomrule
%   \end{tabular}
% \end{table}

% \begin{table}[hbt]
%   \centering
%   \caption{$p$-values from Shapiro-Wilk Test for Oats 102}
%   \label{tab:pvalues_purities_sb231}
%   \begin{tabular}{lrrrrr}
%     \toprule
%     \textbf{Day} & \textbf{16 May} & \textbf{09 June} & \textbf{03 July} & \textbf{27 July} & \textbf{20 Aug.}\\
%                  & \textbf{2016} & \textbf{2016} & \textbf{2016} & \textbf{2016} & \textbf{2016}\\\midrule

%     \textbf{$p$-value} & 0.4928 & $5.24\times 10^{-8}$ & 0.5263 & 0.0484 & 0.9694\\
%     \bottomrule
%   \end{tabular}
% \end{table}


\section{Data analysis}

Descriptive statistics

Fitting distributions

Separability tests


\subsection{Geodesic Scattering Type Angle}

In this analysis, the geodesic distance to the trihedral - the normalized geodesic scattering type angle - was obtained from the samples. From these computed distances, those referring to pixels closer to the trihedral than to the other elementary scatterers (cylinder, dipole, dihedral, narrow dihedral, left helix, right helix, $-1/4$-wave, $+1/4$-wave) were selected.

From this selected distances per image, it was generated the histograms shown in the figure \ref{fig:histograms_alpha_sb231}. Through these graphs, one can suppose that these data obey a Beta distribution. To evaluate this assumption, the Komolgorov-Smirnov test was performed, whose $p$-values are in the table \ref{tab:pvalues_alpha_sb231} along with the sample size.

\begin{figure}[hbt]
\centering
\subcaptionbox{16 May 2016}{\includegraphics[width = \linewidth]{alpha_sb231_1}}
\subcaptionbox{09 June 2016}{\includegraphics[width = \linewidth]{alpha_sb231_2}}
\subcaptionbox{03 July 2016}{\includegraphics[width = \linewidth]{alpha_sb231_3}}
\subcaptionbox{27 July 2016}{\includegraphics[width = \linewidth]{alpha_sb231_4}}
\subcaptionbox{20 August 2016}{\includegraphics[width = \linewidth]{alpha_sb231_5}}
\caption{Histograms of the geodesic distances between trihedral and the pixels from Soybeans 231 most similar to trihedral}
\label{fig:histograms_alpha_sb231}
\end{figure}

\begin{table}[hbt]
  \centering
  \caption{$p$-values from Komolgorov-Smirnov Test for normalized from Soybeans 231}
  \label{tab:pvalues_alpha_sb231}
  \begin{tabular}{lrrrrr}
    \toprule
    \textbf{Day} & \textbf{16 May} & \textbf{09 June} & \textbf{03 July} & \textbf{27 July} & \textbf{20 Aug.}\\ 
                 & \textbf{2016} & \textbf{2016} & \textbf{2016} & \textbf{2016} & \textbf{2016}\\\midrule
    \textbf{Sample size} & 1285 & 656 & 449 & 615 & 508\\
    \textbf{$p$-value} & 0.7746 & 0.5734 & 0.3137 & 0.2392 & 0.4158\\
    \bottomrule
  \end{tabular}
\end{table}

\begin{figure}[hbt]
\centering
\subcaptionbox{16 May 2016}{\includegraphics[width = \linewidth]{alpha_cn43_1}}
\subcaptionbox{09 June 2016}{\includegraphics[width = \linewidth]{alpha_cn43_2}}
\subcaptionbox{03 July 2016}{\includegraphics[width = \linewidth]{alpha_cn43_3}}
\subcaptionbox{27 July 2016}{\includegraphics[width = \linewidth]{alpha_cn43_4}}
\subcaptionbox{20 August 2016}{\includegraphics[width = \linewidth]{alpha_cn43_5}}
\caption{Histograms of the geodesic distances between trihedral and the pixels from Canola 43 most similar to trihedral}
\label{fig:histograms_alpha_cn43}
\end{figure}

\begin{table}[hbt]
  \centering
  \caption{$p$-values from Komolgorov-Smirnov Test for normalized alpha from Canola 43}
  \label{tab:pvalues_alpha_cn43}
  \begin{tabular}{lrrrrr}
    \toprule
    \textbf{Day} & \textbf{16 May} & \textbf{09 June} & \textbf{03 July} & \textbf{27 July} & \textbf{20 Aug.}\\ 
                 & \textbf{2016} & \textbf{2016} & \textbf{2016} & \textbf{2016} & \textbf{2016}\\\midrule
    \textbf{Sample size} & 1549 & 804 & 385 & 133 & 164\\
    \textbf{$p$-value} & 0.2292 & 0.6397 & 0.8759 & 0.3613 & 0.8828\\
    \bottomrule
  \end{tabular}
\end{table}

\begin{figure}[hbt]
\centering
\subcaptionbox{16 May 2016}{\includegraphics[width = \linewidth]{alpha_ot102_1}}
\subcaptionbox{09 June 2016}{\includegraphics[width = \linewidth]{alpha_ot102_2}}
\subcaptionbox{03 July 2016}{\includegraphics[width = \linewidth]{alpha_ot102_3}}
\subcaptionbox{27 July 2016}{\includegraphics[width = \linewidth]{alpha_ot102_4}}
\subcaptionbox{27 July 2016}{\includegraphics[width = \linewidth]{alpha_ot102_5}}
\caption{Histograms of the geodesic distances between trihedral and the pixels from Oats 102 most similar to trihedral}
\label{fig:histograms_alpha_ot102}
\end{figure}

\begin{figure}[hbt]
\centering
\subcaptionbox{16 May 2016}{\includegraphics[width = \linewidth]{/alpha_wt104_1}}
\subcaptionbox{09 June 2016}{\includegraphics[width = \linewidth]{/alpha_wt104_2}}
\subcaptionbox{03 July 2016}{\includegraphics[width = \linewidth]{/alpha_wt104_3}}
\subcaptionbox{27 July 2016}{\includegraphics[width = \linewidth]{/alpha_wt104_4}}
\subcaptionbox{20 August 2016}{\includegraphics[width = \linewidth]{/alpha_wt104_5}}
\caption{Histograms of the geodesic distances between trihedral and the pixels from Wheat 104 most similar to trihedral}
\label{fig:histograms_alpha_wt104}
\end{figure}

\begin{table}[hbt]
  \centering
  \caption{$p$-values from Komolgorov-Smirnov Test for normalized alpha from Oats 102}
  \label{tab:pvalues_alpha_ot102}
  \begin{tabular}{lrrrrr}
    \toprule
    \textbf{Day} & \textbf{16 May} & \textbf{09 June} & \textbf{03 July} & \textbf{27 July} & \textbf{20 Aug.}\\ 
                 & \textbf{2016} & \textbf{2016} & \textbf{2016} & \textbf{2016} & \textbf{2016}\\\midrule
    \textbf{Sample size} & 1207 & 991 & 263 & 82 & 66\\
    \textbf{$p$-value} & 0.3504 & 0.2841 & 0.4707 & 0.2933 & 0.3958\\
    \bottomrule
  \end{tabular}
\end{table}

\begin{table}[hbt]
  \centering
  \caption{$p$-values from Komolgorov-Smirnov Test for normalized alpha from Wheat 104}
  \label{tab:pvalues_alpha_wt104}
  \begin{tabular}{lrrrrr}
    \toprule
    \textbf{Day} & \textbf{16 May} & \textbf{09 June} & \textbf{03 July} & \textbf{27 July} & \textbf{20 Aug.}\\ 
                 & \textbf{2016} & \textbf{2016} & \textbf{2016} & \textbf{2016} & \textbf{2016}\\\midrule
    \textbf{Sample size} & 1382 & 413 & 131 & 122 & 116\\
    \textbf{$p$-value} & 0.1795 & 0.8726 & 0.5488 & 0.3261 & 0.7663\\
    \bottomrule
  \end{tabular}
\end{table}

\begin{table}[hbt]
  \centering
  \caption{Maximum Likelihood estimated Beta parameters for normalized alpha from Soybeans 231}
  \label{tab:params_helicity}
  \begin{tabular}{lrrrrr}
    \toprule
    \textbf{Day} & \textbf{16 May} & \textbf{09 June} & \textbf{03 July} & \textbf{27 July} & \textbf{20 Aug.}\\ 
                 & \textbf{2016} & \textbf{2016} & \textbf{2016} & \textbf{2016} & \textbf{2016}\\\midrule
    \textbf{$\hat{\alpha}$} & 3.1092 & 4.1239 & 3.7793 & 4.6720 & 4.8640\\
    \textbf{$\hat{\beta}$} & 11.5536 & 13.6485 & 11.2156 & 13.8374 & 13.8050\\
    \bottomrule
  \end{tabular}
\end{table}

\begin{table}[hbt]
  \centering
  \caption{Maximum Likelihood estimated Beta parameters for normalized alpha from Canola 43}
  \label{tab:params_alpha_cn43}
  \begin{tabular}{lrrrrr}
    \toprule
    \textbf{Day} & \textbf{16 May} & \textbf{09 June} & \textbf{03 July} & \textbf{27 July} & \textbf{20 Aug.}\\ 
                 & \textbf{2016} & \textbf{2016} & \textbf{2016} & \textbf{2016} & \textbf{2016}\\\midrule
    \textbf{$\hat{\alpha}$} & 2.5812 & 4.0304 & 5.6943 & 6.0239 & 6.0750\\
    \textbf{$\hat{\beta}$} & 10.3476 & 13.4359 & 16.1670 & 14.8314 & 15.2520\\
    \bottomrule
  \end{tabular}
\end{table}

\begin{table}[hbt]
  \centering
  \caption{Maximum Likelihood estimated Beta parameters for normalized alpha from Oats 102}
  \label{tab:params_alpha_ot102}
  \begin{tabular}{lrrrrr}
    \toprule
    \textbf{Day} & \textbf{16 May} & \textbf{09 June} & \textbf{03 July} & \textbf{27 July} & \textbf{20 Aug.}\\ 
                 & \textbf{2016} & \textbf{2016} & \textbf{2016} & \textbf{2016} & \textbf{2016}\\\midrule
    \textbf{$\hat{\alpha}$} & 3.1005  & 2.9157 & 4.9835 & 7.2026 & 6.8227\\
    \textbf{$\hat{\beta}$} & 10.5901 & 10.3148 & 14.9391 & 21.4717 & 18.5422\\
    \bottomrule
  \end{tabular}
\end{table}

\begin{table}[hbt]
  \centering
  \caption{Maximum Likelihood estimated Beta parameters for normalized alpha from Wheat 104}
  \label{tab:params_alpha_wt104}
  \begin{tabular}{lrrrrr}
    \toprule
    \textbf{Day} & \textbf{16 May} & \textbf{09 June} & \textbf{03 July} & \textbf{27 July} & \textbf{20 Aug.}\\ 
                 & \textbf{2016} & \textbf{2016} & \textbf{2016} & \textbf{2016} & \textbf{2016}\\\midrule
    \textbf{$\hat{\alpha}$} & 2.7237 & 4.6521 & 7.8296 & 4.1457 & 5.7762\\
    \textbf{$\hat{\beta}$} & 10.2709 & 12.8761 & 20.2683 & 9.2153 & 14.3660\\
    \bottomrule
  \end{tabular}
\end{table}

Additionally, we performed a separability test based on the Hellinger distance between the original samples assuming the Beta distribution.
Table~\ref{tab:pvalues_sep_alpha_sb231} shows the $p$-values of the null hypothesis that each pair comes from the same law.
We observe that at level $0.05$, the only null hypothesis that cannot be rejected is that the data from the two last dates come from the same law.

\begin{table}[hbt]
  \footnotesize
  \centering
  \caption{$p$-values from Separability Test for normalized alpha from Soybeans 231}
  \label{tab:pvalues_sep_alpha_sb231}
  \begin{tabular}{ccccc}
  \toprule
& \textbf{09 June} & \textbf{03 July} & \textbf{27 July} & \textbf{20 Aug.}\\ \midrule
  \textbf{16 May}  & $7.47 \times 10^{-8}$ & $9.22 \times 10^{-12}$ & $5.16 \times 10^{-21}$ & $1.31 \times 10^{-24}$ \\
  \textbf{09 June}  & --- & $2.64 \times 10^{-3}$ & $4.55 \times 10^{-4}$ & $1.76 \times 10^{-6}$ \\
  \textbf{03 July}  & --- & --- & $4.32 \times 10^{-2}$ & $1.07 \times 10^{-2}$\\
  \textbf{27 July}  & --- & --- & --- & $3.64 \times 10^{-1}$ \\
  \bottomrule
  \end{tabular}
\end{table}

\begin{table}[hbt]
  \footnotesize
  \centering
  \caption{$p$-values from Separability Test for normalized alpha from Canola 43}
  \label{tab:pvalues_sep_alpha_cn43}
  \begin{tabular}{ccccc}
  \toprule
  & \textbf{09 June} & \textbf{03 July} & \textbf{27 July} & \textbf{20 Aug.}\\ \midrule
  \textbf{16 May}  & $1.86 \times 10^{-19}$ & $4.07 \times 10^{-36}$ & $2.75 \times 10^{-25}$ & $9.79 \times 10^{-27}$ \\
  \textbf{09 June}  & --- & $5.28 \times 10^{-8}$ & $2.94 \times 10^{-10}$ & $3.06 \times 10^{-10}$ \\
  \textbf{03 July}  & --- & --- & $1.48 \times 10^{-2}$ & $3.21 \times 10^{-2}$\\
  \textbf{27 July}  & --- & --- & --- & $9.39 \times 10^{-1}$ \\
  \bottomrule
  \end{tabular}
\end{table}

\begin{table}[hbt]
  \footnotesize
  \centering
  \caption{$p$-values from Separability Test for normalized alpha from Oats 102}
  \label{tab:pvalues_sep_alpha_ot102}
  \begin{tabular}{ccccc}
  \toprule
  & \textbf{09 June} & \textbf{03 July} & \textbf{27 July} & \textbf{20 Aug.}\\ \midrule
  \textbf{16 May}  & $2.77 \times 10^{-1}$ & $3.71 \times 10^{-8}$ & $2.18 \times 10^{-7}$ & $7.32 \times 10^{-7}$ \\
  \textbf{09 June}  & --- & $2.08 \times 10^{-10}$ & $1.47 \times 10^{-8}$ & $5.45 \times 10^{-8}$ \\
  \textbf{03 July}  & --- & --- & $1.09 \times 10^{-1}$ & $1.01 \times 10^{-1}$\\
  \textbf{27 July}  & --- & --- & --- & $4.08 \times 10^{-1}$ \\
  \bottomrule
  \end{tabular}
\end{table}

\begin{table}[hbt]
  \footnotesize
  \centering
  \caption{$p$-values from Separability Test for normalized alpha from Wheat 104}
  \label{tab:pvalues_sep_alpha_wt104}
  \begin{tabular}{ccccc}
  \toprule
  & \textbf{09 June} & \textbf{03 July} & \textbf{27 July} & \textbf{20 Aug.}\\ \midrule
  \textbf{16 May}  & $4.24 \times 10^{-27}$ & $2.61 \times 10^{-24}$ & $3.08 \times 10^{-19}$ & $1.32 \times 10^{-17}$ \\
  \textbf{09 June}  & --- & $5.42 \times 10^{-4}$ & $2.70 \times 10^{-4}$ & $5.58 \times 10^{-2}$ \\
  \textbf{03 July}  & --- & --- & $3.31 \times 10^{-5}$ & $1.58 \times 10^{-1}$\\
  \textbf{27 July}  & --- & --- & --- & $3.24 \times 10^{-2}$ \\
  \bottomrule
  \end{tabular}
\end{table}

\subsection{Geodesic Helicity}

%In the figures \ref{fig:histograms_helicity} the histograms of normalized helicities computed for each sample are shown. For these, goodness-of-fit tests were performed -- through Komolgorov-Smirnov Test -- with Beta distribution, whose $p$-values are shown in table \ref{tab:pvalues_helicities}.

% \begin{figure}[hbt]
%     \centering
%     \subcaptionbox{16 May 2016}{\includegraphics[width = \linewidth]{Figures/Soybeans_231/helicity_sb231_1}}
%     \subcaptionbox{09 June 2016}{\includegraphics[width = \linewidth]{Figures/Soybeans_231/helicity_sb231_2}}
%     \subcaptionbox{03 July 2016}{\includegraphics[width = \linewidth]{Figures/Soybeans_231/helicity_sb231_3}}
%     \subcaptionbox{27 July 2016}{\includegraphics[width = \linewidth]{Figures/Soybeans_231/helicity_sb231_4}}
%     \subcaptionbox{20 August 2016}{\includegraphics[width = \linewidth]{Figures/Soybeans_231/helicity_sb231_5}}
%     \caption{Histograms of the normalized geodesic helicities}
%     \label{fig:histograms_helicity}
% \end{figure}

% \begin{table}[hbt]
%   \centering
%   \caption{Maximum Likelihood estimated Beta parameters for normalized helicity from Soybeans 231}
%   \label{tab:params_helicity}
%   \begin{tabular}{lrrrrr}
%     \toprule
%     \textbf{Day} & \textbf{16 May} & \textbf{09 June} & \textbf{03 July} & \textbf{27 July} & \textbf{20 Aug.}\\ 
%                  & \textbf{2016} & \textbf{2016} & \textbf{2016} & \textbf{2016} & \textbf{2016}\\\midrule
%     \textbf{$\hat{\alpha}$} & 2.0984 & 2.4793 & 3.0300 & 2.8211 & 3.2862\\
%     \textbf{$\hat{\beta}$} & 26.0484 & 18.0347 & 17.0150 & 16.9940 & 17.6112\\
%     \bottomrule
%   \end{tabular}
% \end{table}


% \begin{table}[hbt]
%   \centering
%   \caption{$p$-values from Komolgorov-Smirnov Test for normalized helicity from Soybeans 231}
%   \label{tab:pvalues_helicities}
%   \begin{tabular}{lrrrrr}
%     \toprule
%     \textbf{Day} & \textbf{16 May} & \textbf{09 June} & \textbf{03 July} & \textbf{27 July} & \textbf{20 Aug.}\\
%                  & \textbf{2016} & \textbf{2016} & \textbf{2016} & \textbf{2016} & \textbf{2016}\\\midrule
%     \textbf{$p$-value} & 0.0018  & 0.0118 & 0.9812 & 0.2525 & 0.3358\\
%     \bottomrule
%   \end{tabular}
% \end{table}

\section{Conclusion}
%The conclusion goes here. This is the beginning paper to look at~\cite{Ratha2020}.

% if have a single appendix:
%\appendix[Proof of the Zonklar Equations]
% or
%\appendix  % for no appendix heading
% do not use \section anymore after \appendix, only \section*
% is possibly needed

% use appendices with more than one appendix
% then use \section to start each appendix
% you must declare a \section before using any
% \subsection or using \label (\appendices by itself
% starts a section numbered zero.)
%


\section*{Acknowledgment}


The authors would like to thank...

\nocite{ClassificationPolSARGeodesic,AGeneralizedVolumeScatteringModelBasedVegetationIndexfromPolarimetricSARData2019,NovelTechniquesforBuiltupAreaExtractionfromPolarimetricSARImages2019,APolSARScatteringPowerFactorizationFrameworkandNovelRollInvariantParametersBasedUnsupervisedClassificationSchemeUsingaGeodesicDistanceinpress}

\bibliographystyle{IEEEtran}
\bibliography{../../../../Bibliography/references}


% % that's all folks
\end{document}