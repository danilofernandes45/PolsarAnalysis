\documentclass[journal]{IEEEtran}

\usepackage{graphicx}
\graphicspath{{../../../../Figures/GRSL_2020/}{../../../../Figures/GRSL_2020/FactorPlots/}{../../../../Images/GRSL_2020/}{../../../../Figures/Soybeans_231/}}

\usepackage{subcaption}
\captionsetup[table]{font=small,size=smaller,textfont=sc}
\captionsetup[figure]{font=small,size=smaller}

\usepackage{booktabs}
\usepackage[T1]{fontenc}
\usepackage{cite}
\usepackage[cmex10]{amsmath}
\usepackage{dblfloatfix}
\usepackage{url}
\usepackage{color}
\usepackage{bm,bbm}
\usepackage{wasysym}
\usepackage{siunitx}
\DeclareSIUnit\pertenmille{\text{\textpertenthousand}}
\usepackage{multirow,bigstrut}

\DeclareMathOperator{\Tr}{Tr}

\begin{document}
	
	\title{Statistical Properties of Geodesic Roll-Invariant Indexes in PolSAR Data over Crops}
	
	\author{Danilo~Fernandes,
		Debanshu~Ratha,
		Avik~Bhattacharya,~\IEEEmembership{Senior~Member,~IEEE},
		and~Alejandro~C.~Frery,~\IEEEmembership{Senior~Member,~IEEE}% <-this % stops a space
		\thanks{D.\ Fernandes is with the Instituto Federal de Alagoas, Brazil. Email: dfc@laccan.ufal.br}% <-this % stops a space
		\thanks{D.\ Ratha and A.\ Bhattacharya are with the \textit{Centre of Studies in Resources Engineering}
			-- CSRE, Indian Institute of Technology Bombay, Mumbai, India. Email: \{debanshu.ratha;avikb\}@csre.iitb.ac.in}% <-this % stops a space
		\thanks{A.\ C.\ Frery is with the \textit{Laborat\'orio de Computa\c c\~ao Cient\'ifica e An\'alise Num\'erica} -- LaCCAN, 
			Universidade Federal de Alagoas -- Ufal, 
			57072-900 Macei\'o, AL -- Brazil, and the Key Lab of Intelligent Perception and Image Understanding of the Ministry of Education, Xidian University, Xi'an, China. Email: acfrery@laccan.ufal.br}
		\thanks{Manuscript received XX; revised YY.}}
	
	\markboth{IEEE Geoscience and Remote Sensing Letters}%
	{D.\ Fernandes et al.\MakeLowercase{\textit{et al.}}: Statistics Geodesic Distances}
	
	\maketitle
	
	\begin{abstract}
		We make qualitative and quantitative analyses of roll-invariant geodesic features (Purity $P_{\text{GD}}$, Scattering Type Angle $\alpha_{\text{GD}}$, and Helicity $\tau_{\text{GD}}$), as measured on five dates of four different crops.
		After a qualitative analysis, we show that the Lognormal distribution is a good model for $P_{\text{GD}}$, while the Beta distribution explains well both $\alpha_{\text{GD}}$ and $\tau_{\text{GD}}$.
		We then verify that the estimated parameters from each date (and the same crop) are significantly different, with only one exception.
		Finally, we make a temporal analysis of the behavior of these estimates.
	\end{abstract}
	
	\begin{IEEEkeywords}
		Geodesic distance, PolSAR, Kennaugh representation, geodesic roll-invariant indices
	\end{IEEEkeywords}
	
	\IEEEpeerreviewmaketitle
	
	\section{Introduction}
	
	\IEEEPARstart{P}{olarimetric}
	data from extended targets are rich in valuable information.
	An active research venue is the definition of features able to summarize such information leading to processing and analysis techniques.
	The most useful features are the target roll-invariant parameters, i.e., those unaffected by the rotation of the target about the radar line of sight.
	
	Among the most important polarimetric features, one should mention Cloude \& Pottier's $H/\alpha$ decomposition~\cite{CloudePottier:97}, and Touzi's roll-invariant target parameters~\cite{Touzi:TGARS:2007}.
	They lead to successful target scattering classification procedures and geophysical parameter extraction.
	
	More recently, Ratha et al.~\cite{APolSARScatteringPowerFactorizationFrameworkandNovelRollInvariantParametersBasedUnsupervisedClassificationSchemeUsingaGeodesicDistanceinpress}, using the notion of the geodesic distance ($\text{GD}$) on the unit sphere embedded in the $16$-dimensional real (Euclidean) space, obtained a set of roll-invariant parameters with which they built a PolSAR scattering classification scheme. 
	The features derived by measuring the geodesic distances between PolSAR observation and elementary targets were useful in
	change detection~\cite{ChangeDetectionPolSARGeodesicDistanceBetweenScatteringMechanisms},
	unsupervised land-cover classification~\cite{ClassificationPolSARGeodesic}, 
	vegetation monitoring~\cite{AGeneralizedVolumeScatteringModelBasedVegetationIndexfromPolarimetricSARData2019}, and 
	extraction of urban footprint~\cite{NovelTechniquesforBuiltupAreaExtractionfromPolarimetricSARImages2019} using PolSAR images.
	The geodesic distance also provides a radar vegetation index (CpRVI) for compact polarimetric data~\cite{ARadarVegetationIndexforCropMonitoringUsingCompactPolarimetricSARData}. 
	
	However, the nature of these features is not deterministic, since they are extracted from extended targets and, thus, subjected to the influence of speckle.
	Nevertheless, none of those above works has explored the features' statistical properties.
	
	The first step in this direction was presented by Fernandes and Frery~\cite{StatisticalPropertiesofGeodesicDistancesBetweenSamplesandElementaryScatterersinPolSARImagery2019}.
	The authors showed that Beta distributions adequately describe geodesic distances between samples and prototypes.
	Such distances are the basis for computing the geodesic roll-invariant parameters, which, thus, inherit their stochastic nature.
	
	
	
	In this paper, we perform a qualitative analysis of the three geodesic indices on samples from four crops and five dates and show that the Lognormal distribution is an adequate model for Geodesic Purity. 
	In contrast, the Beta distribution explains well both the Geodesic Scattering Type Angle and the Geodesic Helicity.
	
	We then use test statistics based on the Hellinger distance to verify that the estimated parameters from each date (and same crop) are significantly different, with only one exception.
	Finally, we make a temporal analysis of the behavior of these estimates in the parameter space.
	
	\section{Methodology}
	
	\subsection{The Kennaugh Representation}
	
	In PolSAR theory, the $2 \times 2$ complex scattering matrix $\bm S$ has
	complete polarimetric information about backscattering
	from targets:
	$$
	\bm S = \begin{bmatrix}
	S_{\text{HH}} &S_{\text{HV}}\\
	S_{\text{VH}} &S_{\text{VV}}
	\end{bmatrix},
	$$
	where the subscripts $\text{H}$ and $\text{V}$ denote horizontal and vertical
	polarizations, respectively. 
	Following the reciprocity theorem
	in the case of a monostatic radar, 
	$S_{\text{HV}}=S_{\text{VH}}$.
	In such case, we obtain the Pauli scattering vector as
	$$
	\bm k = \frac1{\sqrt{2}}
	\begin{bmatrix}
	S_{\text{HH}} + S_{\text{VV}} 
	& S_{\text{HH}} - S_{\text{VV}} 
	& 2S_{\text{HV}}
	\end{bmatrix}^T,
	$$
	where $T$ denotes transposition.
	
	The Pauli representation is useful for visual analysis (assigning its components to the red, green, and blue channels), and for defining the coherency matrix $\bm T$.
	Consider the availability of $n$ measures over the same type of target, then
	$$
	\bm T = \frac{1}{n} \sum_{i=1}^{n}\bm k_i \bm k_i^{T*},
	$$
	where ``$*$'' denotes the complex conjugate.
	
	Given a scattering matrix $\bm{S}$, the $4 \times 4$ real Kennaugh matrix $\bm{K}$ is defined as~\cite{Pottier09}:
	\begin{equation}
		%\label{coKen}
		\bm{K} = \frac{1}{2}\bm{A}^*(\bm{S} \otimes \bm{S}^*) \bm{A}^{*T}, \quad \bm{A} = \left[
		\begin{array}{cccc}
			1 & 0 & 0 & 1\\
			1 & 0 & 0 & -1\\
			0 & 1 & 1 & 0\\
			0 & j & -j & 0
		\end{array}\right],
	\end{equation}
	where $\otimes$ is the Kronecker product, and  $j = \sqrt{-1}$.
	The Kennaugh matrix can be obtained from the coherency matrix $\bm{T}$~\cite{PolarisationApplicationsRemoteSensing}:
	\begin{equation}
		\label{incoKen}
		\bm{K} =
		\left[\arraycolsep=.7pt
		\begin{array}{cccc}
			\frac{T_{11}+T_{22}+ T_{33}}{2} & \Re(T_{12}) & \Re(T_{13}) & \Im(T_{23})\\
			\Re(T_{12}) & \frac{T_{11}+T_{22}-T_{33}}{2} & \Re(T_{23}) & \Im(T_{13})\\
			\Re(T_{13}) & \Re(T_{23}) & \frac{T_{11}-T_{22}+T_{33}}{2} & - \Im(T_{12})\\
			\Im(T_{23}) & \Im(T_{13}) & - \Im(T_{12}) &\frac{-T_{11}+T_{22}+T_{33}}{2}
		\end{array}\right],
	\end{equation}
	where $\Re(\cdot)$ and $\Im(\cdot)$ denote real and imaginary parts of a complex number. 
	
	The polarimetric nature of the target is preserved under the scaling of measurements. 
	In this light, the unit normalized version of a Kennaugh matrix identifies the scattering class to which the observed Kennaugh matrix belongs.
	
	The elements of $\bm{K}$ matrices lie on $\mathbbm{S}^{15}$, the surface of the unit sphere in $\mathbbm{R}^{16}$. 
	Therefore, the geodesic distance is the natural way to measure the distance between the two $\bm{K}$ matrices and targets. 
	
	The $\text{GD}$ between two $4 \times 4$ real Kennaugh matrices $\bm{K}_1$ and $\bm{K}_2$ is given in~\cite{APolSARScatteringPowerFactorizationFrameworkandNovelRollInvariantParametersBasedUnsupervisedClassificationSchemeUsingaGeodesicDistanceinpress} as
	\begin{equation}
		\text{GD}(\bm{K}_1,\bm{K}_2) =  \frac{2}{\pi} \cos^{-1}\frac{\Tr(\bm{K}_1^T\bm{K}_2)}{\sqrt{\Tr(\bm{K}_1^T\bm{K}_1)}\sqrt{\Tr(\bm{K}_2^T\bm{K}_2)}} ,
		\label{eq:GD_Ken}
	\end{equation}
	where $\Tr$ denotes the trace operator. 
	The factor $2/\pi$ normalizes the distance to $[0,1]$. 
	Therefore, $\text{GD}$ is the distance between the projections of $\bm{K}_1$ and $\bm{K}_2$ on the unit sphere centered at the origin in the space of $4 \times 4$ real matrices. 
	The $\text{GD}$ without the $2/\pi$ factor is an angular parameter (in radians) as it is computed over a unit sphere. 
	The quantity $\text{GD}(\bm{K}_1, \bm{K}_2)$ measures the angle between unit sphere projections of Kennaugh matrices $\bm{K}_1$ and  $\bm{K}_2$.
	
	The GD and associated measures have been successfully employed in several applications~\cite{ClassificationPolSARGeodesic,AGeneralizedVolumeScatteringModelBasedVegetationIndexfromPolarimetricSARData2019,NovelTechniquesforBuiltupAreaExtractionfromPolarimetricSARImages2019,APolSARScatteringPowerFactorizationFrameworkandNovelRollInvariantParametersBasedUnsupervisedClassificationSchemeUsingaGeodesicDistanceinpress,ChangeDetectionPolSARGeodesicDistanceBetweenScatteringMechanisms,ARadarVegetationIndexforCropMonitoringUsingCompactPolarimetricSARData}. Their statistical properties, however, have not been explored yet.
	
	\subsection{Prototypes and Geodesic Measures}
	
	The polarimetric backscattering response of prototypes provides a fundamental reference for the analysis and interpretation of PolSAR data.
	Some of these prototypes are
	{trihedral} ($\text{t}$),
	{cylinder} (${\text{c}}$),
	the Freeman-Durden~\cite{freeman98} random volume (${\text{rv}}$),
	{dipole} (${\text{dp}}$),
	{$+1/4$-wave} (${+1/4}$), 
	{$-1/4$-wave} (${-1/4}$),
	{narrow dihedral} (${\text{nd}}$),
	{dihedral} (${\text{d}}$),
	{left helix} (${\text{lh}}$), 
	{right helix} (${\text{rh}}$), 
	and Ideal Depolarizer (${\text{ID}}$).
	Table~\ref{Tab:ElementaryK} shows the Kennaugh matrices used in this work. 
	Mandal et al.~\cite{ARadarVegetationIndexforCropMonitoringUsingCompactPolarimetricSARData} apply such matrices to crop monitoring using compact polarimetric SAR data, but with a different random volume model.
	
	\begin{table}[hbt]
		\centering
		\caption{Kennaugh matrices of prototypes: {trihedral} ($\bm{K}_{\text{t}}$),
			{cylinder} ($\bm{K}_{\text{c}}$),
			{random volume} ($\bm{K}_{\text{rv}}$),
			{dipole} ($\bm{K}_{\text{dp}}$),
			{$+1/4$-wave} ($\bm{K}_{+1/4}$), 
			{$-1/4$-wave} ($\bm{K}_{-1/4}$),
			{narrow dihedral} ($\bm{K}_{\text{nd}}$),
			{dihedral} ($\bm{K}_{\text{d}}$),
			{left helix} ($\bm{K}_{\text{lh}}$), 
			{right helix} ($\bm{K}_{\text{rh}}$), 
			and Ideal Depolarizer ($\bm{K}_{\text{ID}}$)}\label{Tab:ElementaryK}
		\setlength{\tabcolsep}{2.7pt}
		\renewcommand{\arraystretch}{1.3}
		\begin{tabular}{*{17}{c}}\toprule
			&    \multicolumn{4}{c}{First Row} 
			& \multicolumn{4}{c}{Second Row} 
			& \multicolumn{4}{c}{Third Row} 
			& \multicolumn{4}{c}{Fourth Row}\\ 
			\cmidrule(rl){2-5} \cmidrule(rl){6-9} \cmidrule(rl){10-13} \cmidrule(rl){14-17} 
			$\bm K_{\text{t}}$
			& $1$ & $0$ & $0$ & $0$
			& $0$ & $1$ & $0$ & $0$
			& $0$ & $0$ & $1$ & $0$
			& $0$ & $0$ & $0$ & $-1$ \\
			$\bm K_{\text{c}}$
			& $\frac{5}{8}$ & $\frac{3}{8}$ & $0$ & $0$
			& $\frac{3}{8}$ & $\frac{5}{8}$ & $0$ & $0$
			& $0$ & $0$ & $\frac{1}{2}$ & $0$
			& $0$ & $0$ & $0$ & $-\frac{1}{2}$\\
			$\bm K_{\text{rv}}$ 
			& $1$ & $0$ & $0$ & $0$
			& $0$ & $\frac{1}{2}$ & $0$ & $0$
			& $0$ & $0$ & $\frac{1}{2}$ & $0$
			& $0$ & $0$ & $0$ & $0$\\
			$\bm K_\text{dp}$
			& $1$ & $-1$ & $0$ & $0$
			& $-1$ & $1$ & $0$ & $0$
			& $0$ & $0$ & $0$ & $0$
			& $0$ & $0$ & $0$ & $0$\\
			$\bm K_{+1/4}$
			& $1$ & $0$ & $0$ & $0$
			& $0$ & $1$ & $0$ & $0$
			& $0$ & $0$ & $0$ & $1$
			& $0$ & $0$ & $1$ & $0$\\
			$\bm K_{-1/4}$
			& $1$ & $0$ & $0$ & $0$
			& $0$ & $1$ & $0$ & $0$
			& $0$ & $0$ & $0$ & $-1$
			& $0$ & $0$ & $-1$ & $0$\\
			$\bm K_{\text{nd}}$ 
			&$\frac{5}{8}$ & $\frac{3}{8}$ & $0$ & $0$
			& $\frac{3}{8}$ & $\frac{5}{8}$ & $0$ & $0$
			& $0$ & $0$ & $-\frac{1}{2}$ & $0$
			& $0$ & $0$ & $0$ & $\frac{1}{2}$\\
			$\bm K_{\text{d}}$ &
			$1$ & $0$ & $0$ & $0$
			& $0$ & $1$ & $0$ & $0$
			& $0$ & $0$ & $-1$ & $0$
			& $0$ & $0$ & $0$ & $1$ \\
			$ \bm K_{\text{lh}}$
			& $1$ & $0$ & $0$ & $-1$
			& $0$ & $0$ & $0$ & $0$
			& $0$ & $0$ & $0$ & $0$
			& $-1$ & $0$ & $0$ & $1$\\
			$ \bm K_{\text{rh}}$
			& $1$ & $0$ & $0$ & $1$
			& $0$ & $0$ & $0$ & $0$
			& $0$ & $0$ & $0$ & $0$
			& $1$ & $0$ & $0$ & $1$\\
			$\bm{K}_{\text{ID}}$
			& $1$ & $0$ & $0$ & $0$
			& $0$ & $0$ & $0$ & $0$
			& $0$ & $0$ & $0$ & $0$
			& $0$ & $0$ & $0$ & $0$ \\
			\bottomrule
		\end{tabular}
	\end{table}
	
	Ratha et al.~\cite{APolSARScatteringPowerFactorizationFrameworkandNovelRollInvariantParametersBasedUnsupervisedClassificationSchemeUsingaGeodesicDistanceinpress} proposed three new roll-invariant geodesic parameters, in analogy with classical ones~\cite{gil85,CloudePottier:97,Touzi:TGARS:2007}: Purity, Scattering Type Angle, and Helicity.
	
	The geodesic purity index measures the distance between a target and the ideal depolarizer, which is characterized by the $\bm{K}_{\text{ID}}$ matrix.
	The geodesic purity index is then:
	\begin{equation}
		P_{\text{GD}} = \Big(\frac{3}{2}\text{GD}(\bm{K}, \bm{K}_{\text{ID}})\Big)^2.
	\end{equation}
	
	The geodesic scattering type angle $\alpha_{\text{GD}}$ describes 
	the distance to the trihedral prototype:
	\begin{equation}
		\alpha_{\text{GD}}'(\bm{K}) = \frac{\pi}{2}  \text{GD}(\bm{K},\bm{K}_{\text{t}}).
	\end{equation}
	
	The helicity parameter provides a quantitative estimation of target symmetry~\cite{Touzi:TGARS:2007} in the observation. 
	This quantity is derived as the (geometric) mean of the distances from left and right helices:
	\begin{equation}
		\tau_{\text{GD}}' = \frac\pi4 \big(1 - \sqrt{\text{GD}(\bm{K},\bm{K}_{\text{lh}})\text{GD}(\bm{K},\bm{K}_{\text{rh}})}\big).
	\end{equation}
	The multiplication by $\pi/4$ makes the scale equal (in radians) for comparison with $|\tau_{m_1}|$~\cite{Touzi:TGARS:2007}. 
	For trihedral target, $\tau_{\text{GD}}' = 0$ and for helices $\tau_{\text{GD}}' = \pi/4$, which matches with the scattering at extremities of $|\tau_{m_1}|$.
	
	We will analyze the geodesic scattering type angle and the geodesic purity index without scaling, i.e., $\alpha_{\text{GD}}(\bm{K}) = \text{GD}(\bm{K},\bm{K}_{\text{t}})$, and 
	$\tau_{\text{GD}} = 1 - \sqrt{\text{GD}(\bm{K},\bm{K}_{\text{lh}})\text{GD}(\bm{K},\bm{K}_{\text{rh}})}$.
	With this, both measures lie in $[0,1]$.
	
	\section{Statistical Description}
	
	\subsection{Initial considerations}
	
	Fernandes and Frery~\cite{StatisticalPropertiesofGeodesicDistancesBetweenSamplesandElementaryScatterersinPolSARImagery2019} showed that Beta distributions describe geodesic distances between carefully selected samples and prototypes.
	They did not analyze the GD to the Ideal Depolarizer and, thus, there is no available information about the distribution of $P_{\text{GD}}$.
	
	With those results in mind, on the one hand, one might expect that $\alpha_{\text{GD}}$ follows a Beta distribution.
	On the other hand, the distribution of $\sqrt{XY}$, with $X$ and $Y$ Beta random variables, cannot be expressed, in general, in a simple form.
	If these random variables are independent, which is unlikely our case, the distribution of their product involves the ${}_2F_{1}$ hypergeometric function~\cite[Corollary~2.1]{OntheDistributionoftheProductofIndependentBetaRandomVariables}.
	Therefore, there is also no available information about the distribution of $\tau_{\text{GD}}$.
	
	\subsection{Samples}
	
	We analyzed five PolSAR images of the same region: the image from 16 May 2016, followed by four others at time intervals of $24$ days. 
	The images are $30 \times 65$ pixels.
	Figures~\ref{fig:day_0} to~\ref{fig:day_96} show the Canola data in Pauli Decomposition. 
	
	\begin{figure}[hbt]
		\centering
		\subcaptionbox{16 May\label{fig:day_0}}{\includegraphics[width = .19\linewidth]{sb231_day_0}}
		\subcaptionbox{9 June\label{fig:day_24}}{\includegraphics[width = .19\linewidth]{sb231_day_24}}
		\subcaptionbox{3 July\label{fig:day_48}}{\includegraphics[width = .19\linewidth]{sb231_day_48}}
		\subcaptionbox{27 July\label{fig:day_72}}{\includegraphics[width = .19\linewidth]{sb231_day_72}}
		\subcaptionbox{20 Aug.\label{fig:day_96}}{\includegraphics[width = .19\linewidth]{sb231_day_96}}
		\caption{Canola samples taken in 2016, in Pauli decomposition}
		\label{fig:sample_images}
	\end{figure}
	
	Fig.~\ref{fig:AllIndexes} shows estimates of the densities of the three Geodesic Indices (Purity $P_{\text{GD}}$, Scattering Type Angle $\alpha_{\text{GD}}$, and Helicity $\tau_{\text{GD}}$) for the four crops (Canola, Oats, Soy Beans, and Wheat) in the five dates considered here.
	
	\begin{figure}[htb]
		\centering
		\includegraphics[width=\linewidth]{Indexes}
		\caption{Density estimates of the three Geodesic Indices by crop and date.}\label{fig:AllIndexes}
	\end{figure}
	
	\subsection{Data Analysis}
	
	\subsubsection{Geodesic Purity}
	
	An exploratory data analysis revealed that the logarithmic transformation of the Geodesic Purity $P_{\text{GD}}$ leads to data that can be well described by the Gaussian distribution.
	We, thus, opted for fitting the Lognormal law, characterized by the probability density function
	\begin{equation}
		f(x;\mu,\sigma^2) = \frac{1}{\sqrt{2\pi\sigma^2} x} \exp\Big\{
		-\frac1{2 \sigma^2}\big(\log x - \mu\big)^2
		\Big\},
	\end{equation}
	where $\mu\in\mathbbm R$, and $\sigma^2>0$ are the parameters.
	
	Table~\ref{tab:params_purity} shows the Maximum Likelihood estimates of the Lognormal model ($\widehat \mu,\widehat{\sigma^2}$), and $p$-value of the Anderson-Darling goodness-of-fit test for the Geodesic Purity $P_{\text{GD}}$.
	All samples sizes are $n=1950$.
	Notice that the minimum $p$-value is $0.076$; there is, thus, no evidence that the Lognormal models should not be used for these samples.
	The Lognormal distribution has the additional advantage that it describes data which, after a logarithmic transformation, can be treated as Gaussian.
	
	Fig.~\ref{fig:WorstPurityLognormalFit} shows the histogram of the data with the worst fit (Wheat, 16~May), and the fitted density; we used the Freedman-Diaconis rule to compute the number of bins.
	
	\begin{table}[hbt]
		\centering
		\caption{Maximum Likelihood estimates of the Lognormal model ($\widehat \mu,\widehat{\sigma^2}$), and $p$-value of the Anderson-Darling goodness-of-fit test for the Geodesic Purity $P_{\text{GD}}$.}
		\label{tab:params_purity}
		\setlength{\tabcolsep}{2pt}
		\begin{tabular}{clrrrrr}
			\toprule
			\textbf{Crop $\backslash$ Date} & & \textbf{16 May} & \textbf{9 June} & \textbf{3 July} & \textbf{27 July} & \textbf{20 Aug.}\\ \cmidrule(lr){1-2} \cmidrule(lr){3-3} \cmidrule(lr){4-4} \cmidrule(lr){5-5} \cmidrule(lr){6-6} \cmidrule(lr){7-7}
			\textbf{Canola}     
			& $\widehat{\mu}$         & $-0.0890$  & $-0.1738$   & $-0.2681$   & $-0.3187$   & $-0.3072$ \\
			& $\widehat{\sigma^2}$     & 0.0295     & 0.0676     & 0.0994     & 0.1269      & 0.1361 \\ 
			& $p$-value             & 0.2701    & 0.2739     & 0.7064     & 0.5727     & 0.7515\\        
			\midrule
			\textbf{Soy Beans}
			& $\widehat{\mu}$         & $-0.1120$    & $-0.1831$    & $-0.2356$ & $-0.2278$ & $-0.2350$         \\
			& $\widehat{\sigma^2}$     & 0.0334       & 0.0670       & 0.0859       & 0.079       & 0.0848 \\ 
			& $p$-value             & 0.8921     & 0.8273     & 0.7992     & 0.9841     & 0.7099\\            
			\midrule
			\textbf{Oats}
			& $\widehat{\mu}$         & $-0.1223$ & $-0.1358$    & $-0.2578$ & $-0.2679$    & $-0.3008$ \\
			& $\widehat{\sigma^2}$     & 0.0479     & 0.0737       & 0.1085     & 0.1267      & 0.1338 \\ 
			& $p$-value             & 0.8438     & 0.1024     & 0.8262     & 0.7027    & 0.3977 \\
			\midrule
			\textbf{Wheat} 
			& $\widehat{\mu}$         & $-0.0947$    & $-0.2392$ & $-0.2866$ & $-0.2837$ & $-0.3145$ \\
			& $\widehat{\sigma^2}$     & 0.0310       & 0.0972       & 0.1099       & 0.1461       & 0.1269   \\
			& $p$-value             & 0.0760     & 0.5490     & 0.3051     & 0.4034     & 0.4177\\    
			\bottomrule
		\end{tabular}
	\end{table}
	
	\begin{figure*}[hbt]
		\centering
		\subcaptionbox{Geodesic Purity $P_{\text{GD}}$ from Wheat, 16~May, and fitted Lognormal density.\label{fig:WorstPurityLognormalFit}}{\includegraphics[width=.25\linewidth]{WheatPurityLognormalFit}}
		%
		\subcaptionbox{Geodesic Scattering Type Angle $\alpha_{\text{GD}}$ from Canola, 16~May, and fitted Beta density.\label{fig:WorstAlphaBetaFit}}{\includegraphics[width=.25\linewidth]{CanolaAlphaBetaFit}}
		%
		\subcaptionbox{Geodesic Helicity $\tau_{\text{GD}}$ from Canola, 16~May, and fitted Beta density.\label{fig:WorstHelicityBetaFit}}{\includegraphics[width=.25\linewidth]{CanolaHelicityBetaFit}}
		\caption{The three worst fits of Geodesic Indices.}
	\end{figure*}
	
	\subsubsection{Geodesic Scattering Type Angle and Helicity}
	
	The observations are bounded in $[0,1]$, 
	and the shape of their estimated densities also suggests the use of the Beta distribution, characterized by the probability density function:
	\begin{equation}
		f(x; p, q) = \frac{\Gamma(p+q)}{\Gamma(p)\Gamma(q)} x^{p-1} (1-x)^{q-1} \mathbbm 1_{(0,1)}(x),
		\label{eq:BetaDensity}
	\end{equation}
	with $p,q>0$.
	
	For $\alpha_{\text{GD}}$, we computed the distance of each pixel to all prototypes (except to $\bm K_{\text{ID}}$), and retained those closer to $\bm K_{\text{t}}$ than to the others.
	
	Table~\ref{tab:params_alpha} shows, for each crop type and date, 
	the sample size ($n$), 
	the Maximum Likelihood estimates of the Beta model ($\widehat p,\widehat q$), and the $p$-value of the Kolmogorov-Smirnoff goodness-of-fit test for the Geodesic Scattering Type Angle $\alpha_{\text{GD}}$.
	Notice that, although the minimum $p$-value is $0.0103$, the others range between $0.0723$ and $0.9887$; there is, thus, no evidence that the Beta models should not be used for these samples.
	Fig.~\ref{fig:WorstAlphaBetaFit} shows the histogram of the data with the worst fit (Canola, 16~May), and the fitted density; we used the Freedman-Diaconis rule to compute the number of bins.
	
	\begin{table}[hbt]
		\centering
		\caption{Sample size ($n$), Maximum Likelihood estimates of the Beta model ($\widehat p,\widehat q$), and $p$-value of the Kolmogorov-Smirnof goodness-of-fit test for the Geodesic Scattering Type Angle $\alpha_{\text{GD}}$.}
		\label{tab:params_alpha}
		\setlength{\tabcolsep}{3.8pt}
		\begin{tabular}{clrrrrr}
			\toprule
			\textbf{Crop $\backslash$ Date} & & \textbf{16 May} & \textbf{9 June} & \textbf{3 July} & \textbf{27 July} & \textbf{20 Aug.}\\ \cmidrule(lr){1-2} \cmidrule(lr){3-3} \cmidrule(lr){4-4} \cmidrule(lr){5-5} \cmidrule(lr){6-6} \cmidrule(lr){7-7}
			\textbf{Canola}     & $n$             & 1389         & 730         & 333         & 130         & 145\\
			& $\widehat{p}$ & 2.7560     & 4.1886     & 5.8105     & 6.1547     & 6.3503\\
			& $\widehat{q}$ & 11.0141     & 13.9375     & 16.4850     & 15.1480     & 15.9220\\ 
			& $p$-value     & 0.0103     & 0.2955     & 0.9887     & 0.9503     & 0.6245\\        
			\midrule
			\textbf{Soy Beans}    & $n$             & 1285         & 656         & 449         & 615         & 508\\
			& $\widehat{p}$ & 3.2348     & 4.3843     & 4.0188     & 4.8295     & 5.1401\\
			& $\widehat{q}$ & 11.9960     & 14.4563     & 11.8697     & 14.2715     & 14.5409\\ 
			& $p$-value     & 0.5666     & 0.4052     & 0.1814     & 0.2687     & 0.3187\\            
			\midrule
			\textbf{Oats}    & $n$             & 1207         & 991         & 263         & 82         & 66\\
			& $\widehat{p}$ & 3.2822      & 3.1221     & 5.2792     & 8.0549     & 7.5261\\
			& $\widehat{q}$ & 11.1652     & 10.9882     & 15.7778     & 23.9306     & 20.3871\\ 
			& $p$-value     & 0.1300     & 0.1526     & 0.4645     & 0.1992     & 0.3646\\
			\midrule
			\textbf{Wheat}     & $n$             & 1382         & 413         & 131         & 122         & 116\\
			& $\widehat{p}$ & 2.9096      & 4.8869     & 8.4877     & 4.5926     & 6.1919\\
			& $\widehat{q}$ & 10.9248     & 13.4887     & 21.9157     & 10.1156     & 15.3390\\
			& $p$-value     & 0.0723     & 0.7890     & 0.3638     & 0.2067     & 0.7346\\    
			\bottomrule
		\end{tabular}
	\end{table}
	
	Table~\ref{tab:params_helicity} shows the Maximum Likelihood estimates of the Beta model ($\widehat p,\widehat q$), and the $p$-value of the Kolmogorov-Smirnoff goodness-of-fit test for the Geodesic Helicity $\tau_{\text{GD}}$.
	All samples sizes are $n=1950$.
	Although the $p$-value of five out of twenty samples falls below \SI{1}{\percent}, they are all larger than \SI{1}{\pertenmille}.
	We, thus, have no substantial evidence against the Beta model for describing the Geodesic Helicity.
	Fig.~\ref{fig:WorstHelicityBetaFit} shows the histogram of the data with the worst fit (Canola, 16~May), and the fitted density; we used the Freedman-Diaconis rule to compute the number of bins.
	
	\begin{table}[hbt]
		\centering
		\caption{Maximum Likelihood estimates of the Beta model ($\widehat p,\widehat q$), and $p$-value of the Kolmogorov-Smirnof goodness-of-fit test for the Geodesic Helicity $\tau_{\text{GD}}$.}
		\label{tab:params_helicity}
		\setlength{\tabcolsep}{3.8pt}
		\begin{tabular}{clrrrrr}
			\toprule
			\textbf{Crop $\backslash$ Date} & & \textbf{16 May} & \textbf{9 June} & \textbf{3 July} & \textbf{27 July} & \textbf{20 Aug.}\\ \cmidrule(lr){1-2} \cmidrule(lr){3-3} \cmidrule(lr){4-4} \cmidrule(lr){5-5} \cmidrule(lr){6-6} \cmidrule(lr){7-7}
			\textbf{Canola}     
			& $\widehat{p}$ & 2.1856      & 2.6837    & 5.8105     & 3.3877      & 4.4111 \\
			& $\widehat{q}$ & 32.5176     & 20.3264     & 16.4850     & 15.5491     & 14.9766\\ 
			& $p$-value     & 1E$-4$         & 0.0645     & 0.9887     & 0.3356     & 0.0166\\        
			\midrule
			\textbf{Soy Beans}
			& $\widehat{p}$ & 2.4317       & 2.7304       & 3.0602       & 2.9543       & 3.4111 \\
			& $\widehat{q}$ & 30.1012   & 19.8095   & 17.1821   & 17.7749   & 18.2627 \\ 
			& $p$-value     & 5E$-4$     & 0.005     & 0.9388     & 0.0514     & 0.1081\\            
			\midrule
			\textbf{Oats}
			& $\widehat{p}$ & 2.3701       & 2.1838      & 3.529     & 4.2011     & 4.7568 \\
			& $\widehat{q}$ & 26.5330     & 20.4121   & 16.4932     & 16.1089      & 15.9517 \\ 
			& $p$-value     & 0.0016     & 1E$-4$     & 0.541     & 0.0078    & 0.1757 \\
			\midrule
			\textbf{Wheat} 
			& $\widehat{p}$ & 2.3516       & 3.3384      & 4.3960      & 4.1996      & 4.6823   \\
			& $\widehat{q}$ & 33.4388   & 17.1858   & 16.6094   & 14.8613   & 15.2931   \\
			& $p$-value     & 5E$-4$     & 0.2118     & 0.0476     & 0.0215     & 1E$-4$\\    
			\bottomrule
		\end{tabular}
	\end{table}
	
	\section{Temporal Evolution}
	
	We first checked the hypothesis that every pair of parameters from the same crop in different dates comes from different models.
	We used test statistics based on the Hellinger distance between the models~\cite{AnalyticExpressionsStochasticDistancesBetweenRelaxedComplexWishartDistributions}, and found that all the pairs are identifiable with \SI{5}{\percent} of significance, with the sole exception of $P_{\text{GD}}$ from Soy Beans in the two last dates.
	
	Fig.~\ref{fig:TemporalIndexes} (top) shows the temporal evolution of the parameter estimates $(\widehat\mu,\widehat\sigma)$ of the Lognormal distribution when describing the Geodesic Purity index $P_{\text{GD}}$.
	Canola, Oats, Soy Beans, and Wheat share similar behavior: $\widehat\mu$ decreases, while $\widehat\sigma$ increases with time, except for the two last dates of Canola and Wheat.
	The five stages of these four crops are easily identifiable using the Lognormal model, whereas Soy Beans have only four identifiable stages since, as noted, the estimates from the two last dates cannot be considered different.
	
	In-situ measurements of the SMAPVEX16 campaign confirm that majority of the wheat and oat fields were at tillering during the second week of June. The Manitoba weekly crop reports~\cite{manitobaagriculture} suggest that soybean seeding finished during this period, and the plants were in their unifoliate to third trifoliate growth stage indicating very low Plant Area Index (PAI) and biomass values.
	
	From the Manitoba weekly crop reports~\cite{manitobaagriculture}, canola seeding was almost completed, and the crop was emerging rapidly. 
	Earlier seeded fields were reaching the rosette stage. 
	Thus, the effect of the soil is expected to dominate the radar response rather than the vegetation component.
	
	Fig.~\ref{fig:TemporalIndexes} (middle) shows the temporal evolution of the parameter estimates $(\widehat p, \widehat q)$ of the Beta distribution when modeling the Geodesic Scattering Type $\alpha_{\text{GD}}$.
	Wheat is the crops where the estimates span most of the parametric space, as opposed to Soy Beans that gather around small values.
	Although all estimates are significantly different, the two first dates of Oats are very close.
	
	During the first week of July, cereals (wheat and oats) were at their stem elongation to flag leaf stages. Some fields advanced to the heading stage. Consequently, we observe increases in PAI and biomass. 
	In contrast, increases in PAI and wet biomass for soybean are less. According to the Manitoba weekly crop reports~\cite{manitobaagriculture}, the majority of canola fields were bolting, forming early flowers, whereas podding started in the most advanced fields.
	
	The estimates $(\widehat p, \widehat q)$ of the Beta distribution for the Geodesic Helicity $\tau_{\text{GD}}$ (Fig.~\ref{fig:TemporalIndexes} bottom) suggest a decreasing tendency of $\widehat q$ along with an increasing trend of $\widehat p$ along time.
	
	\begin{figure}
		\centering
		\includegraphics[width=\columnwidth]{TemporalIndexes}
		\caption{Temporal evolution of the parameter estimates $(\widehat\mu,\widehat\sigma)$ of the Lognormal distribution for the Geodesic Purity ($P_{\text{GD}}$, top), and of $(\widehat p,\widehat q)$ of the Beta distribution for the Geodesic Scattering Type Angle ($\alpha_{\text{GD}}$, middle) and Helicity ($\tau_{\text{GD}}$, bottom).}\label{fig:TemporalIndexes}
	\end{figure}
	
	\section{Conclusion}
	
	Refs.~\cite{ClassificationPolSARGeodesic,AGeneralizedVolumeScatteringModelBasedVegetationIndexfromPolarimetricSARData2019,NovelTechniquesforBuiltupAreaExtractionfromPolarimetricSARImages2019,APolSARScatteringPowerFactorizationFrameworkandNovelRollInvariantParametersBasedUnsupervisedClassificationSchemeUsingaGeodesicDistanceinpress,ChangeDetectionPolSARGeodesicDistanceBetweenScatteringMechanisms,ARadarVegetationIndexforCropMonitoringUsingCompactPolarimetricSARData}
	have shown the expressiveness and usefulness of indices derived from Geodesic Distances, but with no assessment of their statistical properties.
	This paper shows that the three main Geodesic Indices, namely
	Geodesic Purity ($P_{\text{GD}}$),
	Geodesic Scattering Type Angle ($\alpha_{\text{GD}}$), and
	Geodesic Helicity ($\tau_{\text{GD}}$) can be described by parsimonious and easy to use models.
	
	The Lognormal distribution well describes Geodesic Purity.
	This implies that the logarithmic transformation $\log P_{\text{GD}}$ can be assumed to follow a Gaussian distribution.
	The Beta distribution can model both the Geodesic Scattering Type Angle and the Geodesic Helicity.
	These two models are indexed by two real parameters, with techniques readily available in most statistical software.
	
	We used \texttt{R}~\cite{RManual} v.~3.6.3 for data processing and analysis (in particular, we used the functions provided by the \texttt{Rfast} library~\cite{Rfast} for parameter estimation), as well as for producing the graphical outputs (with the \texttt{ggplot2} library~\cite{ggplot2}).
	
	\section*{Acknowledgment}
	
	The authors would like to thank CNPq and Fapeal for providing partial support to this research. 
	D.\ Ratha would like to acknowledge the support of CSIR, Government of India, towards his doctoral studies. The authors would like to thanks Mr.\ Dipankar Mandal for providing information about the crop growth stages from the Manitoba (Canada) Agriculture reports.
	
	\bibliographystyle{IEEEtran}
	\bibliography{references}
	
\end{document}